\documentclass[a4paper,10pt]{scrartcl}

\usepackage[utf8]{inputenc}
\usepackage[ngerman]{babel}
\usepackage[T1]{fontenc}
\usepackage{amsmath}

\title{Mein erstes Dokument}
\author{Max Mustermann}
\date{08.12.2021}

\begin{document}

\maketitle
\tableofcontents

\section{Einleitung}
Hier kommt die Einleitung. Ihre Überschrift kommt automatisch in das Inhaltsverzeichnis. Das Inhaltsverzeichnis wird erst beim zweiten compilieren Angezeigt.

\subsection{Formeln}
\LaTeX{} ist auch ohne Formeln sehr nützlich und eunfach zu verwenden. Grafiken, Tabellen, Querverweise aller Art, Literatur- und Stichwortverweise sind kein Problem.

Formeln sind etwas schwieriger, dennoch hier ein Einfaches Beispiel. Zwei von Einsteins berühmtesten Formeln lauten:
\begin{align}
E &= mc^2
m &= \frac{m_0}{\sqrt{1-\frac{v^2}{c^2}}}
\end{align}
Aber wer keine Formeln schreibt, braucht sich damit auch nicht zu beschäftigen.

\end{document}

